\documentclass[11pt,a4paper]{article}
%以下为所使用的宏包
\usepackage{ulem}%下划线
\usepackage{amsmath,amsfonts,amssymb,amsthm,amsbsy}%数学符号
\usepackage{graphicx}%插入图片
\usepackage{booktabs}%三线表
%\usepackage{indentfirst}%首行缩进
\usepackage{tikz}%作图
\usepackage{appendix}%附录
\usepackage{array}%多行公式/数组
\usepackage{makecell}%表格缩并
\usepackage{siunitx}%SI单位--\SI{number}{unit}
\usepackage{mathrsfs}%数学字体
\usepackage{enumitem}%列表间距
\usepackage{multirow}%列表横向合并单元格
\usepackage[colorlinks,linkcolor=red,anchorcolor=blue,citecolor=green]{hyperref}%超链接引用
\usepackage{float}%图片、表格位置排版
\usepackage{pict2e,keyval,fp,diagbox}%带有斜线的表格
\usepackage{fancyvrb,listings}%设置代码插入环境
\usepackage{minted}%代码环境设置
\usepackage{fontspec}%字体设置
\usepackage{color,xcolor}%颜色设置
\usepackage{titlesec} %自定义标题格式
\usepackage{tabularx}%列表扩展
\usepackage{authblk}%titlepage作者信息
\usepackage{nicematrix}%更好的矩阵标定
\usepackage{fbox}%更多浮动体盒子



%以下是页边距设置
\usepackage[left=0.5in,right=0.5in,top=0.81in,bottom=0.8in]{geometry}

%以下是段行设置
\linespread{1.4}%行距
\setlength{\parskip}{0.1\baselineskip}%段距
\setlength{\parindent}{2em}%缩进


%其他设置
\numberwithin{equation}{section}%公式按照章节编号
\newenvironment{point}{\raggedright$\blacktriangleright$}{}
\newenvironment{algorithm}[1]{\vspace{12pt} \hrule\hrule \vspace{3pt} \noindent\textbf{\color[HTML]{E63F00}Algorithm } \,\textit{#1} \vspace{3pt} \hrule\vspace{6pt}}{\vspace{6pt}\hrule\hrule \vspace{12pt}} % 算法伪代码格式环境


%代码环境\lst设置
\definecolor{CodeBlue}{HTML}{268BD2}
\definecolor{CodeBlue2}{HTML}{0000CD}
\definecolor{CodeGreen}{HTML}{2AA1A2}
\definecolor{CodeRed}{HTML}{CB4B16}
\definecolor{CodeYellow}{HTML}{B58900}
\definecolor{CodePurPle}{HTML}{D33682}
\definecolor{CodeGreen2}{HTML}{859900}
\lstset{
    basicstyle=\tt,%字体设置
    numbers=left, %设置行号位置
    numberstyle=\tiny\color{black}, %设置行号大小
    keywordstyle=\color{black}, %设置关键字颜色
    stringstyle=\color{CodeRed}, %设置字符串颜色
    commentstyle=\color{CodeGreen}, %设置注释颜色
    frame=single, %设置边框格式
    escapeinside=`, %逃逸字符(1左面的键),用于显示中文
    %breaklines, %自动折行
    extendedchars=false, %解决代码跨页时,章节标题,页眉等汉字不显示的问题
    xleftmargin=2em,xrightmargin=2em, aboveskip=1em, %设置边距
    tabsize=4, %设置tab空格数
    showspaces=false, %不显示空格
    emph={TRUE,FALSE,NULL,NAN,NA,<-,},emphstyle=\color{CodeBlue2}, %其他高亮}
}


%节标题格式设置
\titleformat{\section}[block]{\large\bfseries}{Exercise \arabic{section}}{1em}{}[]
\titleformat{\subsection}[block]{}{    \arabic{section}.(\alph{subsection})}{1em}{}[]
% \titleformat{\subsubsection}[block]{\normalsize\bfseries}{    \arabic{subsection}-\alph{subsubsection}}{1em}{}[]
% \titleformat{\paragraph}[block]{\small\bfseries}{[\arabic{paragraph}]}{1em}{}[]


% \titleformat{\sectioncommand}[shape]{format}{title-label}{sep}{before-title}[after-title]



% 中文字号
% 初号42pt, 小初36pt, 一号26pt, 小一24pt, 二号22pt, 小二18pt, 三号16pt, 小三15pt, 四号14pt, 小四12pt, 五号10.5pt, 小五9pt



\begin{document}

\begin{center}\thispagestyle{plain}

{\LARGE\textbf{STAT 430-1, Fall 2024}}

{\Large\textbf{HW1}}

Tuorui Peng\footnote{TuoruiPeng2028@u.northwestern.edu}
\end{center}

\thispagestyle{myheadings}\markright{Compiled using \LaTeX}
\pagestyle{myheadings}\markright{Tuorui Peng}

% show only sections level in contents
\setcounter{tocdepth}{1}
\tableofcontents


  


\section{Intersections and Unions of $ \sigma  $-algebras}

\subsection{}
Say we have a family of $ \sigma $-algebras indexed by $ \mathcal{I} $: $ \{(E,\mathcal{E}_i)\}_{i\in \mathcal{I}} $. Then we verify that $ \bigcap_{i\in \mathcal{I}} \mathcal{E}_i $ is a $ \sigma $-algebra.

\begin{itemize}[topsep=2pt,itemsep=0pt]
    \item \textbf{Empty set}: Since $ \forall i\in \mathcal{I} $ we have $ \emptyset, E \in \mathcal{E}_i $, we have $ \emptyset, E\in \bigcap_{i\in \mathcal{I}} \mathcal{E}_i $.
    \item \textbf{Closed under complement}: For any $ A\in \bigcap_{i\in \mathcal{I}} \mathcal{E}_i $, we have $ A\in \mathcal{E}_i, \forall i\in \mathcal{I} $. Thus $ A^c\in \mathcal{E}_i, \forall i\in \mathcal{I} $, which implies $ A^c\in \bigcap_{i\in \mathcal{I}} \mathcal{E}_i $.
    \item \textbf{Closed under $ \sigma  $-union}: For any $ \{A_n\}_{n\in \mathbb{N}}\subseteq \bigcap_{i\in \mathcal{I}} \mathcal{E}_i $, we have $ A_n\in \mathcal{E}_i, \forall i\in \mathcal{I} $. Thus $ \bigcup_{n\in \mathbb{N}} A_n\in \mathcal{E}_i, \forall i\in \mathcal{I} $, which implies $ \bigcup_{n\in \mathbb{N}} A_n\in \bigcap_{i\in \mathcal{I}} \mathcal{E}_i $.
\end{itemize}
By verifying the three properties, we conclude that $ \bigcap_{i\in \mathcal{I}} \mathcal{E}_i $ is a $ \sigma $-algebra.


\subsection{}

We construct the following example for union of $ \sigma $-algebras:
\begin{align*}
    E=&\{1,2,3\},\\
    \mathcal{E}_1=&\{\emptyset, \{1\}, \{2,3\}, E\},\\
    \mathcal{E}_2=&\{\emptyset, \{1,2\}, \{3\}, E\}.
\end{align*}
Then we have $ \mathcal{E}_1\cup \mathcal{E}_2=\{\emptyset, \{1\}, \{2,3\}, \{1,2\}, \{3\}, E\} $, which is not a $ \sigma $-algebra. For example, $ \{1\}\cup \{3\}=\{1,3\}\notin \mathcal{E}_1\cup \mathcal{E}_2 $ (not closed under union).

Actually the $ \sigma  $-algebra generated by $ \mathcal{E}_1\cup \mathcal{E}_2 $ is 
\begin{align*}
    \mathcal{E}_1\wedge \mathcal{E}_2=&\{\emptyset, \{1\}, \{2\}, \{3\}, \{1,2\}, \{1,3\}, \{2,3\}, E\}.
\end{align*}

\section{Increasing algebras and $ \sigma  $-algebras}

\subsection{}

If $ \mathcal{E}_n $ is algebra, we verify the three properties of algebra for $ \mathcal{E}=\bigcup_{n=1}^\infty \mathcal{E}_n $:
\begin{itemize}[topsep=2pt,itemsep=0pt]
    \item \textbf{Empty set}: Since $ \emptyset\in \mathcal{E}_n, \forall n\in \mathbb{N} $, we have $ \emptyset\in \mathcal{E} $.
    \item \textbf{Closed under complement}: For any $ A\in \mathcal{E} $, meaning that $ \exists n_0\in \mathbb{N}^+ $ s.t. $ A\in \mathcal{E}_n $, $ \forall n\geq n_0 $. Then since $ \mathcal{E}_n $ is algebra, we have $ A^c\in \mathcal{E}_n, \forall n\geq n_0 $, which implies $ A^c\in \mathcal{E} = \bigcup_{n=1}^\infty \mathcal{E}_n $.
    \item \textbf{Cloed under finite union}: It suffices to verify closure under union of 2. For $ A,B\in \mathcal{E} $, we have $ \exists n_1,n_2\in \mathbb{N}^+ $ s.t. $ A\in \mathcal{E}_{n_1}, B\in \mathcal{E}_{n_2} $. Without loss of generality, we assume $ n_1\leq n_2 $. Then we have $ A,B\in \mathcal{E}_{n_2} $, then using closure under finite union of $ \mathcal{E}_{n_2} $, we have $ A\cup B\in \mathcal{E}_{n}, \forall n\geq n_2 $. This implies $ A\cup B\in \mathcal{E} $.
\end{itemize}

Thus we conclude that $ \mathcal{E}=\bigcup_{n=1}^\infty \mathcal{E}_n $ is an algebra.

\subsection{}


Consider the following example:
\begin{align*}
    \mathcal{E}_n = \sigma (\emptyset, \{1\}, \{2\}, \cdots, \{n\}, \mathbb{N}^+)
\end{align*}\
for which all with $ \Omega =\mathbb{N}^+ $. However we find that $ \bigcup_{n=1}^\infty \mathcal{E}_n $ is not a $ \sigma $-algebra by noticing the following:
\begin{align*}
    \{2i\} \in  \mathcal{E}=\bigcup_{n=1}^\infty \mathcal{E}_n,\,\forall i\in\mathbb{R},\text{ but } \bigcup_{i\in \mathbb{N}^+} \{2i\} = \{2,4,6,\cdots\}\notin \mathcal{E}.
\end{align*}
which implies that $ \mathcal{E} $ is not closed under countable union. Thus $ \bigcup_{n=1}^\infty \mathcal{E}_n $ is not a $ \sigma $-algebra.




\section{Borel $ \sigma  $-algebra on $ \mathbb{R} $}

\subsection{}\label{openintervals}

\begin{itemize}[topsep=2pt,itemsep=0pt]
    \item Any $ \sigma $-algebra containing all open sets must contain all open intervals, which is trivial. (And definitely we have $ \mathcal{B}_\mathbb{R} $ being a $ \sigma $-algebra, and it contains all open intervals.)
    \item Then consider any $ \sigma $-algebra $ \mathcal{A} $ containing all open intervals. In this case, for any open sets $ O \subseteq \mathbb{R} $, by definition, $ \exists $ intervals $ \{I_i\}_{i\in\mathbb{N}^+} $ s.t.
    \begin{align*}
        O=\bigcup_{i\in \mathbb{N}^+} I_i. 
    \end{align*}
    On the other hand, since $ \mathcal{A} $ contains all open intervals, i.e. $ I_i\in \mathcal{A}, \forall i\in \mathbb{N}^+ $. Thus $ O = \bigcup_{i\in \mathbb{N}^+} I_i\in \mathcal{A} $, which implies that $ \mathcal{A} $ contains all open sets. 
\end{itemize}

By summarizing the two points, we conclude that: any $ \sigma $-algebra containing all open sets must also contain all open intervals, and vice versa. Thus the Borel $ \sigma $-algebra $ \mathcal{B}_\mathbb{R} $, as the intersection of all $ \sigma $-algebras containing all open sets, is also the smallest $ \sigma $-algebra containing all open intervals.


\subsection{}

We verify the following cases:
\begin{itemize}[topsep=2pt,itemsep=0pt]
    \item $ \mathbf{(-\infty, x)} $: Since $ (-\infty, x) $ is open, we have $ (-\infty, x)\in \mathcal{B}_{\mathbb{R}} $.
    \item $ \mathbf{(-\infty,x]} $: We can write $ (-\infty,x] = \bigcap_{n=1}^\infty (-\infty,x+\frac{1}{n}) $, thus $ (-\infty,x]\in \mathcal{B}_{\mathbb{R}} $.

    \item $ \mathbf{(x,y]}$: We can write $ (x,y] = (-\infty,y] \cap (x,y+1) $, thus $ (x,y]\in \mathcal{B}_{\mathbb{R}} $.
    \item $ \mathbf{[x,y]} $: Similarly, we can write $ [x,y] = (-\infty,y] \cap [x,+\infty) $, thus $ [x,y]\in \mathcal{B}_{\mathbb{R}} $.
    \item $ \mathbf{\{x\}} $: We can write $ \{x\} = (-\infty,x] \cap [x,+\infty) $, thus $ \{x\}\in \mathcal{B}_{\mathbb{R}} $.
\end{itemize}

\subsection{}

We first prove the following lemma:

\begin{point}
    For a collection of intervals $ G=\{g_i\}_{i\in\mathcal{I}} $ (could be uncountable). If any open set $ O\subseteq \mathbb{R} $ can be written as \textbf{countable} operations of $ g_i $ (operations include union, intersection, and complement), then $ \mathcal{B}_\mathbb{R} $ is also the generated $ \sigma $-algebra by $ G $.
\end{point}

\begin{proof}
    The proof is similar to that in \ref{openintervals}.
    \begin{itemize}[topsep=2pt,itemsep=0pt]
        \item Since $ G $ is a collection of intervals, then any $ \sigma  $-algebra containing all open sets must also contain all elements in $ G $.
        \item On the other hand, for any $ \sigma  $-algebra $ \mathcal{A} $ containing all elements in $ G $, then for any open set $ O\subseteq \mathbb{R} $, by assumption, $ O $ can be written as a countable operation of $ g_i $, i.e. $ O\in \mathcal{A} $. While $ g_i\in  \mathcal{A} $ gives that this countable operation is also in $ \mathcal{A} $. Thus $ O\in \mathcal{A} $, which implies that $ \mathcal{A} $ contains all open sets.
    \end{itemize}
    By summarizing the two points, we conclude that $ \mathcal{B}_\mathbb{R} $ is also the generated $ \sigma $-algebra by $ G $.
\end{proof}


Then it suffices to show that in the four cases listed in the questions, any open sets can be written as countable operations of the intervals. Further since open sets are countable unions of open intervals, \textbf{it suffices to show that open intervals can be written as countable operations of the intervals given in eeach of the four cases.}

\begin{itemize}[topsep=2pt,itemsep=0pt]
    \item[(i)] Any $ (a,b) = \big( (-\infty ,a]^\complement \big) \cap \big( \bigcup_{i=1}^\infty (-\infty, b-\dfrac{ 1 }{ i } ] \big) $
    \item[(ii)] Any $ (a,b) =  \bigcup_{i=1}^\infty (a, b-\dfrac{ 1 }{ i } ] $
    \item[(iii)] Any $ (a,b) =  \bigcup_{i=1}^\infty [a+\dfrac{ 1 }{ i }, b - \dfrac{ 1 }{ i } ]$
    \item[(iv)] Any $ (a,b) =  \big((a,\infty)\big)\cap \big( \bigcup_{i=1}^\infty (b-\dfrac{ 1 }{ i },\infty )^\complement \big)$
\end{itemize}

Further, to upgrade to rational number cases, we may just notice that any real number $ x $ could be written as a monotone (say, increasing) limit of rational number sequence $ \{x_n\}_{n\in \mathbb{N}^+} $\, $ x_n\in \mathbb{Q} $ (for example, using decimal numeral system). In this way, we can obtain any interval with real ends by countable operations of intervals with rational ends. Thus the Borel $ \sigma $-algebra $ \mathcal{B}_\mathbb{R} $ is generated by the four cases listed in the question.


\section{Borel $ \sigma  $-algebra and continuous functions}

We have the following two points:
\begin{itemize}[topsep=2pt,itemsep=0pt]
    \item Consider any $ \sigma  $-algebra $ \tilde{\mathcal{E}} $ that makes all continuous functions measurable. $  \tilde{\mathcal{E}}  \subseteq \mathcal{B}_{\mathbb{R}^n}$.
    
    \begin{proof}
        For any continuous function $ f:(\mathbb{R}^n, \mathcal{B}_{\mathbb{R}^n})\mapsto (\mathbb{R}, \mathcal{B}_{\mathbb{R}}) $, we can check for
        \begin{align*}
            f^{-1}((-\infty, x)) = \{\omega \in \mathbb{R}^n: f(\omega )< x\} \in \mathcal{B}_{\mathbb{R}^n}
        \end{align*}
        which is obvious form the definition of continuous functions. Thus we have $  \tilde{\mathcal{E}}  \subseteq \mathcal{B}_{\mathbb{R}^n}$.
    \end{proof}

    So further, denote $ \mathcal{E} $ the minimal $ \sigma  $-algebra that makes all continuous functions measurable, we have also $  \mathcal{E}  \subseteq \mathcal{B}_{\mathbb{R}^n} $.

    \item If a $ \sigma  $-algebra $ \mathcal{E} $ is the smallest one that makes all continuous functions measurable, then $ \mathcal{B}_{\mathbb{R}^n} \subseteq  \mathcal{E}$.
    
    \begin{proof}
        It suffices to show that for such $ \mathcal{E} $, it has to contain all closed sets. So consider any given closed set $ C\subseteq \mathbb{R}^n $ and a corresponding continuous function $ f_C(\omega ) $ defined as the distance of $ \omega  $ to $ C $, i.e. $ f_C(\omega ) = \mathop{ \inf  }\limits_{x\in C} \|\omega -x\| $. Then $ f_C $ is continuous, and thus $ f_C^{-1}(\{0\}) = C\in \mathcal{E} $. Such construction is possible for any closed set $ C\in\mathbb{R}^n $, so $ \mathcal{E} $ contains all closed sets, while these closed sets generate $ \mathcal{B}_{\mathbb{R}^n} $. So we have $ \mathcal{B}_{\mathbb{R}^n} \subseteq  \mathcal{E} $.
    \end{proof}
\end{itemize}
    
    Combining the above two points, we conclude that $ \mathcal{B}_{\mathbb{R}^n} \subseteq \mathcal{E} \subseteq \mathcal{B}_{\mathbb{R}^n} $, which implies that $ \mathcal{E} = \mathcal{B}_{\mathbb{R}^n} $. i.e. the Borel $ \sigma $-algebra on $ \mathbb{R}^n $ is the smallest $ \sigma $-algebra that makes all continuous functions measurable.












\section{Restrictions and traces of measures}

\subsection{}
We verify the conditions of a measure:
\begin{itemize}[topsep=2pt,itemsep=0pt]
    \item \textbf{Empty set}: Since $ \mu $ is a measure, we have $ \nu(\emptyset)=\mu (\emptyset \cap D) = \mu(\emptyset) =0 $
    \item \textbf{Countable additivity}: For any disjoint sets $ \{A_n\}_{n\in \mathbb{N}}\subseteq \mathcal{E}$, we have also $ \{A_n\cap D\}_{n\in \mathbb{N}}\subseteq \mathcal{E} $ being disjoint. Then
    \begin{align*}
        \nu(\biguplus_{n\in \mathbb{N}} A_n) = \mu\left( \left( \biguplus_{n\in \mathbb{N}} A_n \right) \cap D \right) = \mu\left( \biguplus_{n\in \mathbb{N}} (A_n\cap D) \right) = \sum_{n\in \mathbb{N}} \mu(A_n\cap D) = \sum_{n\in \mathbb{N}} \nu(A_n).
    \end{align*}   
    
\end{itemize}

Thus $ \nu $ is a measure on $ (E,\mathcal{E}) $.

\subsection{}

We verify the conditions of a measure:
\begin{itemize}[topsep=2pt,itemsep=0pt]
    \item \textbf{Empty set}: Since $ \mathcal{D}=\mathcal{E}\cap D $, thus $ \emptyset \in \mathcal{D} $. We have $ \nu(\emptyset) = \mu (\emptyset )=0 $.
    \item \textbf{Countable additivity}: For any disjoint sets $ \{A_n\}_{n\in \mathbb{N}}\subseteq \mathcal{D} \subseteq \mathcal{E}$, we thus have $ \nu (\biguplus_{n\in \mathbb{N}} A_n) = \mu \left( \biguplus_{n\in \mathbb{N}} A_n \right) = \sum_{n\in \mathbb{N}} \mu(A_n) = \sum_{n\in \mathbb{N}} \nu(A_n) $.
\end{itemize}

Thus $ \nu $ is a measure on $ (D,\mathcal{D}) $.

    
\titleformat{\section}[block]{\large\bfseries}{Appendix}{1em}{}[]
\section{References}

\begin{itemize}[topsep=2pt,itemsep=0pt]
    \item I discussed Question 2.(b) with Yikun Li (yikunli2028@u.northwestern.edu).
    \item For Question 4, I referred to the following StackExchange post: \href{https://math.stackexchange.com/questions/189567/prove-that-borel-sigma-field-on-rd-is-the-smallest-sigma-field-that-makes-all}{StackExchange Post}.
    \item Wikipedia pages on \href{https://en.wikipedia.org/wiki/Measurable_function}{measurable functions} and \href{https://en.wikipedia.org/wiki/Borel_set}{Borel sets}.
    \item Textbooks, including Durrett and Billingsley.
\end{itemize}
    





    




    




\end{document}