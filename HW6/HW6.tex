\documentclass[11pt,a4paper]{article}
%以下为所使用的宏包
\usepackage{ulem}%下划线
\usepackage{amsmath,amsfonts,amssymb,amsthm,amsbsy}%数学符号
\usepackage{graphicx}%插入图片
\usepackage{booktabs}%三线表
%\usepackage{indentfirst}%首行缩进
\usepackage{tikz}%作图
\usepackage{appendix}%附录
\usepackage{array}%多行公式/数组
\usepackage{makecell}%表格缩并
\usepackage{siunitx}%SI单位--\SI{number}{unit}
\usepackage{mathrsfs}%数学字体
\usepackage{enumitem}%列表间距
\usepackage{multirow}%列表横向合并单元格
\usepackage[colorlinks,linkcolor=red,anchorcolor=blue,citecolor=green]{hyperref}%超链接引用
\usepackage{float}%图片、表格位置排版
\usepackage{pict2e,keyval,fp,diagbox}%带有斜线的表格
\usepackage{fancyvrb,listings}%设置代码插入环境
\usepackage{minted}%代码环境设置
\usepackage{fontspec}%字体设置
\usepackage{color,xcolor}%颜色设置
\usepackage{titlesec} %自定义标题格式
\usepackage{tabularx}%列表扩展
\usepackage{authblk}%titlepage作者信息
\usepackage{nicematrix}%更好的矩阵标定
\usepackage{fbox}%更多浮动体盒子



%以下是页边距设置
\usepackage[left=0.5in,right=0.5in,top=0.81in,bottom=0.8in]{geometry}

%以下是段行设置
\linespread{1.4}%行距
\setlength{\parskip}{0.1\baselineskip}%段距
\setlength{\parindent}{2em}%缩进


%其他设置
\numberwithin{equation}{section}%公式按照章节编号
\newenvironment{point}{\raggedright$\blacktriangleright$}{}
\newenvironment{algorithm}[1]{\vspace{12pt} \hrule\hrule \vspace{3pt} \noindent\textbf{\color[HTML]{E63F00}Algorithm } \,\textit{#1} \vspace{3pt} \hrule\vspace{6pt}}{\vspace{6pt}\hrule\hrule \vspace{12pt}} % 算法伪代码格式环境


%代码环境\lst设置
\definecolor{CodeBlue}{HTML}{268BD2}
\definecolor{CodeBlue2}{HTML}{0000CD}
\definecolor{CodeGreen}{HTML}{2AA1A2}
\definecolor{CodeRed}{HTML}{CB4B16}
\definecolor{CodeYellow}{HTML}{B58900}
\definecolor{CodePurPle}{HTML}{D33682}
\definecolor{CodeGreen2}{HTML}{859900}
\lstset{
    basicstyle=\tt,%字体设置
    numbers=left, %设置行号位置
    numberstyle=\tiny\color{black}, %设置行号大小
    keywordstyle=\color{black}, %设置关键字颜色
    stringstyle=\color{CodeRed}, %设置字符串颜色
    commentstyle=\color{CodeGreen}, %设置注释颜色
    frame=single, %设置边框格式
    escapeinside=`, %逃逸字符(1左面的键),用于显示中文
    %breaklines, %自动折行
    extendedchars=false, %解决代码跨页时,章节标题,页眉等汉字不显示的问题
    xleftmargin=2em,xrightmargin=2em, aboveskip=1em, %设置边距
    tabsize=4, %设置tab空格数
    showspaces=false, %不显示空格
    emph={TRUE,FALSE,NULL,NAN,NA,<-,},emphstyle=\color{CodeBlue2}, %其他高亮}
}


%节标题格式设置
\titleformat{\section}[block]{\large\bfseries}{Exercise \arabic{section}}{1em}{}[]
\titleformat{\subsection}[block]{}{    \arabic{section}.(\alph{subsection})}{1em}{}[]
% \titleformat{\subsubsection}[block]{\normalsize\bfseries}{    \arabic{subsection}-\alph{subsubsection}}{1em}{}[]
% \titleformat{\paragraph}[block]{\small\bfseries}{[\arabic{paragraph}]}{1em}{}[]


% \titleformat{\sectioncommand}[shape]{format}{title-label}{sep}{before-title}[after-title]



% 中文字号
% 初号42pt, 小初36pt, 一号26pt, 小一24pt, 二号22pt, 小二18pt, 三号16pt, 小三15pt, 四号14pt, 小四12pt, 五号10.5pt, 小五9pt



\begin{document}

\begin{center}\thispagestyle{plain}

{\LARGE\textbf{STAT 430-1, Fall 2024}}

{\Large\textbf{HW6}}

Tuorui Peng\footnote{TuoruiPeng2028@u.northwestern.edu}
\end{center}

\thispagestyle{myheadings}\markright{Compiled using \LaTeX}
\pagestyle{myheadings}\markright{Tuorui Peng}

% show only sections level in contents
\setcounter{tocdepth}{1}
\tableofcontents

  

% \section{Kronecker Lemma}

% The condition $ \sum_{k=1}^\infty \dfrac{ a_k }{ k }<\infty  $ is equivalent to that: $ \forall  \varepsilon >0 $, there exists $ N $ such that $ \forall M>N $, $ \sum\limits_{k=N}^M \dfrac{ a_k }{ k }<\varepsilon  $. Then we have:
% \begin{align*}
%     \dfrac{ 1 }{ M  } \sum_{k=1}^M  a_k \leq & \dfrac{ 1 }{ M  } \sum_{k=1}^{N-1}  a_k + \sum_{k=N}^M  \dfrac{ a_k }{ k } \leq \dfrac{ 1 }{ M  } \sum_{k=1}^N  a_k + \varepsilon
% \end{align*}
% then by noticing that $ \dfrac{ 1 }{ M  } \sum\limits_{k=1}^{N-1}  a_k \xrightarrow[]{M\to\infty}  0 $, we are able to control the left side up to any precision. Thus we have
% \begin{align*}
%     \dfrac{ 1 }{ M  } \sum_{k=1}^M  a_k \to 0 ,\quad M\to\infty
% \end{align*}



\vspace{12pt}


\hrule
\vspace{12pt}

\textbf{Notation:} I use $ \fallingdotseq $ for Fourier transform and $ \risingdotseq $ for inverse Fourier transform. i.e. $ f(x)\fallingdotseq \phi (t) $ and $ \phi (t)\risingdotseq f(x) $. $ \mathrm{ Res }_f(x)  $ for the residue of $ f(x) $ at $ x $. Dirac delta function (at zero) is denoted as $ \delta (x) $ s.t. $ \int_{-\varepsilon }^\varepsilon \delta (x) \,\mathrm{d}x=1 ,\,\forall \varepsilon >0 $.



\section{Self-normalized sum}

\begin{itemize}[topsep=2pt,itemsep=0pt]
    \item \textbf{Denominator:} By SLLN and continuous mapping theorem, we have
    \begin{align*}
        \sqrt{\dfrac{ \sum_{i=1}^n X_i^2 }{ n } } \xrightarrow[]{\mathrm{a.s.}} \sqrt{ \mathbb{E} X_1^2 } = \sqrt{1/3}
    \end{align*}
    \item \textbf{Numerator:} By CLT, we have
    \begin{align*}
        \dfrac{ \sum_{i=1}^n X_i }{ \sqrt{n}\cdot var(X_1) } = \dfrac{ \sum_{i=1}^n X_i }{ \sqrt{n}\sqrt{1/3} } \xrightarrow[]{\mathcal{D}} N(0,1)
    \end{align*}
\end{itemize}
Combining the two, and by Slutsky's theorem, we have
\begin{align*}
    \dfrac{ \sum_{i=1}^n X_i }{ \sqrt{\sum_i=1}^n X_i^2 } = & \sqrt{\dfrac{ 1 }{ 3 } }\dfrac{ \sum_{i=1}^n X_i \big/ \sqrt{n}\sqrt{1/3} }{ \sqrt{\sum_{i=1}^n X_i^2 \big/ n } }   \xrightarrow[]{\mathrm{d}} N(0,1)
\end{align*}


\section{Geometric mean}

\subsection{}

It suffices to show that $\log G_n = \dfrac{ 1 }{ n }\sum_{i=1}^n \log X_i \xrightarrow[]{\mathrm{a.s.}} -1   $ by continuous mapping theorem. 

To show that we notice that with $ X_i\mathop{ \sim  }\limits^{i.i.d.} \mathrm{ Unif }(0,1)   $, we have $ -\log X_i \mathop{ \sim }\limits^{i.i.d.}\mathrm{ Exp }(1)   $, thus we have by SLLN
\begin{align*}
    \dfrac{ 1 }{ n }\sum_{i=1}^n \log X_i \xrightarrow[]{\mathrm{a.s.}} -\mathbb{E}\left[ \mathrm{ Exp }(1)  \right] = -1
\end{align*}
thus finishes the proof that $ G_n\xrightarrow[]{\mathrm{a.s.}} e^{-1} $.

\subsection{}
Still use the log transformation, we have
\begin{align*}
    \log (eG_n)^{\sqrt{n}} = \sqrt{n}\left( 1+\dfrac{ 1 }{ n }\sum_{i=1}^n \log X_i \right) \xrightarrow[(1)]{\mathrm{d}} N(0,1)
\end{align*}
in which (1) is by CLT by noticing that $ \mathbb{E}\left[ \log X_1 \right] =-1 $ and $ \mathrm{Var}\left[ \log X_1 \right] =1 $. Thus by continuous mapping theorem applied to $ \xi \mapsto \exp \xi  $, we have
\begin{align*}
    (eG_n)^{\sqrt{n}} \xrightarrow[]{\mathrm{d}} \exp N(0,1) \sim \mathrm{ LogNormal }(0,1) 
\end{align*}
in which $ \mathrm{ LogNormal }(0,1)  $ is the \href{https://en.wikipedia.org/wiki/Log-normal_distribution}{log-normal distribution}, with density function
\begin{align*}
    f_\mathrm{ LogNormal(0,1) }(x) = \dfrac{ 1 }{ x\sqrt{2\pi} }\exp\left( -\dfrac{ (\log x)^2 }{ 2 } \right)
\end{align*}

\section{Weak convergence for finitely supported distributions}

\begin{itemize}[topsep=2pt,itemsep=0pt]
    \item["$  \Rightarrow  $"] Given that $ \mathbb{P}\left( X_n=k \right)\xrightarrow[]{n\to\infty} \mathbb{P}\left( X=k \right),\,\forall k\in S  $, we have also that $ \forall k\in S $:
    \begin{align*}
        \mathbb{P}\left( X_n\leq k \right) = & \sum_{i=1}^k \mathbb{P}\left( X_n=i \right) \xrightarrow[]{n\to\infty} \sum_{i=1}^k \mathbb{P}\left( X=i \right) = \mathbb{P}\left( X\leq k \right) 
    \end{align*}
    thus we have $ X_n\xrightarrow[]{\mathrm{d}}  X $.
    \item["$  \Leftarrow  $"] Given that $ X_n\xrightarrow[]{\mathrm{d}}  X $, we have that $ \forall k\in S $:
    \begin{align*}
        \mathbb{P}\left( X_n=k \right) = & \mathbb{P}\left( X_n\leq k \right) - \mathbb{P}\left( X_n\leq k-1 \right) \xrightarrow[]{n\to\infty} \mathbb{P}\left( X\leq k \right) - \mathbb{P}\left( X\leq k-1 \right) = \mathbb{P}\left( X=k \right)
    \end{align*}
    (with a trivial fix for $ k=0 $ that $ \mathbb{P}\left( X_n=0 \right)\to\mathbb{P}\left( X=0 \right)  $ automatically holds). Thus we have $ \mathbb{P}\left( X_n=k \right)\xrightarrow[]{n\to\infty} \mathbb{P}\left( X=k \right),\,\forall k\in S  $.

\end{itemize}


\section{Coupling Poisson distributions}

Note that for independent Poisson random variables $ W,Z $ with parameters $ \nu _1,\nu _2 $, we have $ W+Z\sim \mathrm{ Poisson }(\nu _1+\nu _2)  $. The proof is as follows:
\begin{proof}
    Using characteristic function, we have
    \begin{align*}
        W+Z\fallingdotseq & \exp\left( \nu _1(e^{it}-1) \right) \exp\left( \nu _2(e^{it}-1) \right) = \exp\left( (\nu _1+\nu _2)(e^{it}-1) \right) \risingdotseq \mathrm{ Poisson }(\nu _1+\nu _2) .
    \end{align*}    
\end{proof}

Thus for the Poisson distributed random variables $ X,Y $ with parameter $ \lambda ,\mu  $, respectively, we can construct a $ \delta \sim \mathrm{ Poisson }(\mu -\lambda ) \geq 0  $ and by the above property we have a coupling of $ X,Y $ that 
\begin{align*}
    Y' = X'+\delta \sim \mathrm{ Poisson }(\lambda +\mu ),\qquad X'\sim \mathrm{ Poisson }(\lambda ),\qquad \delta \sim \mathrm{ Poisson }(\mu -\lambda ) \geq 0.
\end{align*}
and thus proves that $ Y $ first-order stochastically dominates $ X $.


    


\section{Coupling Exponential distributions}
Using log transform, we have that 
\begin{align*}
    \mathrm{ Exp }(\lambda ) \sim -\dfrac{ 1 }{ \lambda  }  \log \mathrm{ Unif }(0,1) 
\end{align*}
thus we can construct a coupling of exponential random variables $ U,V $ with parameters $ \lambda ,\mu  $, respectively, that
\begin{align*}
    V'= -\dfrac{ \mu  }{ 1 }\log e^{-\lambda U'} = \dfrac{ \lambda  }{ \mu  }U'  \sim \mathrm{ Exp }(\mu )   ,\quad U'\sim \mathrm{ Exp }(\lambda )
\end{align*}
in which we notice that $ 0<\lambda <\mu  \Leftrightarrow V'< U' $, thus we have $ V $ first-order stochastically dominates $ U $.


















    







\end{document}