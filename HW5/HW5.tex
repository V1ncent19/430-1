\documentclass[11pt,a4paper]{article}
%以下为所使用的宏包
\usepackage{ulem}%下划线
\usepackage{amsmath,amsfonts,amssymb,amsthm,amsbsy}%数学符号
\usepackage{graphicx}%插入图片
\usepackage{booktabs}%三线表
%\usepackage{indentfirst}%首行缩进
\usepackage{tikz}%作图
\usepackage{appendix}%附录
\usepackage{array}%多行公式/数组
\usepackage{makecell}%表格缩并
\usepackage{siunitx}%SI单位--\SI{number}{unit}
\usepackage{mathrsfs}%数学字体
\usepackage{enumitem}%列表间距
\usepackage{multirow}%列表横向合并单元格
\usepackage[colorlinks,linkcolor=red,anchorcolor=blue,citecolor=green]{hyperref}%超链接引用
\usepackage{float}%图片、表格位置排版
\usepackage{pict2e,keyval,fp,diagbox}%带有斜线的表格
\usepackage{fancyvrb,listings}%设置代码插入环境
\usepackage{minted}%代码环境设置
\usepackage{fontspec}%字体设置
\usepackage{color,xcolor}%颜色设置
\usepackage{titlesec} %自定义标题格式
\usepackage{tabularx}%列表扩展
\usepackage{authblk}%titlepage作者信息
\usepackage{nicematrix}%更好的矩阵标定
\usepackage{fbox}%更多浮动体盒子



%以下是页边距设置
\usepackage[left=0.5in,right=0.5in,top=0.81in,bottom=0.8in]{geometry}

%以下是段行设置
\linespread{1.4}%行距
\setlength{\parskip}{0.1\baselineskip}%段距
\setlength{\parindent}{2em}%缩进


%其他设置
\numberwithin{equation}{section}%公式按照章节编号
\newenvironment{point}{\raggedright$\blacktriangleright$}{}
\newenvironment{algorithm}[1]{\vspace{12pt} \hrule\hrule \vspace{3pt} \noindent\textbf{\color[HTML]{E63F00}Algorithm } \,\textit{#1} \vspace{3pt} \hrule\vspace{6pt}}{\vspace{6pt}\hrule\hrule \vspace{12pt}} % 算法伪代码格式环境


%代码环境\lst设置
\definecolor{CodeBlue}{HTML}{268BD2}
\definecolor{CodeBlue2}{HTML}{0000CD}
\definecolor{CodeGreen}{HTML}{2AA1A2}
\definecolor{CodeRed}{HTML}{CB4B16}
\definecolor{CodeYellow}{HTML}{B58900}
\definecolor{CodePurPle}{HTML}{D33682}
\definecolor{CodeGreen2}{HTML}{859900}
\lstset{
    basicstyle=\tt,%字体设置
    numbers=left, %设置行号位置
    numberstyle=\tiny\color{black}, %设置行号大小
    keywordstyle=\color{black}, %设置关键字颜色
    stringstyle=\color{CodeRed}, %设置字符串颜色
    commentstyle=\color{CodeGreen}, %设置注释颜色
    frame=single, %设置边框格式
    escapeinside=`, %逃逸字符(1左面的键),用于显示中文
    %breaklines, %自动折行
    extendedchars=false, %解决代码跨页时,章节标题,页眉等汉字不显示的问题
    xleftmargin=2em,xrightmargin=2em, aboveskip=1em, %设置边距
    tabsize=4, %设置tab空格数
    showspaces=false, %不显示空格
    emph={TRUE,FALSE,NULL,NAN,NA,<-,},emphstyle=\color{CodeBlue2}, %其他高亮}
}


%节标题格式设置
\titleformat{\section}[block]{\large\bfseries}{Exercise \arabic{section}}{1em}{}[]
\titleformat{\subsection}[block]{}{    \arabic{section}.(\alph{subsection})}{1em}{}[]
% \titleformat{\subsubsection}[block]{\normalsize\bfseries}{    \arabic{subsection}-\alph{subsubsection}}{1em}{}[]
% \titleformat{\paragraph}[block]{\small\bfseries}{[\arabic{paragraph}]}{1em}{}[]


% \titleformat{\sectioncommand}[shape]{format}{title-label}{sep}{before-title}[after-title]



% 中文字号
% 初号42pt, 小初36pt, 一号26pt, 小一24pt, 二号22pt, 小二18pt, 三号16pt, 小三15pt, 四号14pt, 小四12pt, 五号10.5pt, 小五9pt



\begin{document}

\begin{center}\thispagestyle{plain}

{\LARGE\textbf{STAT 430-1, Fall 2024}}

{\Large\textbf{HW5}}

Tuorui Peng\footnote{TuoruiPeng2028@u.northwestern.edu}
\end{center}

\thispagestyle{myheadings}\markright{Compiled using \LaTeX}
\pagestyle{myheadings}\markright{Tuorui Peng}

% show only sections level in contents
\setcounter{tocdepth}{1}
\tableofcontents

  

% \section{Kronecker Lemma}

% The condition $ \sum_{k=1}^\infty \dfrac{ a_k }{ k }<\infty  $ is equivalent to that: $ \forall  \varepsilon >0 $, there exists $ N $ such that $ \forall M>N $, $ \sum\limits_{k=N}^M \dfrac{ a_k }{ k }<\varepsilon  $. Then we have:
% \begin{align*}
%     \dfrac{ 1 }{ M  } \sum_{k=1}^M  a_k \leq & \dfrac{ 1 }{ M  } \sum_{k=1}^{N-1}  a_k + \sum_{k=N}^M  \dfrac{ a_k }{ k } \leq \dfrac{ 1 }{ M  } \sum_{k=1}^N  a_k + \varepsilon
% \end{align*}
% then by noticing that $ \dfrac{ 1 }{ M  } \sum\limits_{k=1}^{N-1}  a_k \xrightarrow[]{M\to\infty}  0 $, we are able to control the left side up to any precision. Thus we have
% \begin{align*}
%     \dfrac{ 1 }{ M  } \sum_{k=1}^M  a_k \to 0 ,\quad M\to\infty
% \end{align*}



\vspace{12pt}


\hrule
\vspace{12pt}

\textbf{Notation:} I use $ \fallingdotseq $ for Fourier transform and $ \risingdotseq $ for inverse Fourier transform. i.e. $ f(x)\fallingdotseq \phi (t) $ and $ \phi (t)\risingdotseq f(x) $. $ \mathrm{ Res }_f(x)  $ for the residue of $ f(x) $ at $ x $. Dirac delta function (at zero) is denoted as $ \delta (x) $ s.t. $ \int_{-\infty}^\infty \delta (x) \,\mathrm{d}x=1 $.



\section{Characteristic functions}

\begin{itemize}[topsep=2pt,itemsep=0pt]
    \item[(a)] Note that function $ t\mapsto 1/(1+t^2) $ has inverse Fourier transform as follows:
    \begin{align*}
        1/(1+t^2)\risingdotseq & \dfrac{ 1 }{ 2\pi } \int _\mathbb{R} \dfrac{ 1 }{ 1+t^2 }e^{-ixt}  \,\mathrm{d}t =\begin{cases}
            \dfrac{ 2\pi i }{ 2\pi } \mathrm{ Res }_{e^{-ixt}/(1+t^2)}(i),& x>0\\
            \dfrac{ 2\pi i }{ 2\pi } \mathrm{ Res }_{e^{-ixt}/(1+t^2)}(-i),& x\leq 0
        \end{cases}= e^{-|x|}/2
    \end{align*}
    which is the density function of a two-sided exponential distribution.
    \item[(b)] Not a characteristic function by noticing that the characteristic function $ t\mapsto e^{-t^4} :=\phi_X(t)$ yields a $ 0 $ second moment:
    \begin{align*}
        \mathbb{E}\left[ X^2 \right] = & \phi_X''(0) = \dfrac{ \mathrm{d}^2 }{ \mathrm{d}t^2 }e^{-t^4}\big|_{t=0}=0
    \end{align*}
    which means that $ X $ is actually a degenerate r.v. with $ X=0 $ a.s., which should actually correspond to a characteristic function of $ t\mapsto 1 $ instead. Thus the contradiction arises, so $ t\mapsto e^{-t^4} $ is not a characteristic function.

    \textbf{Reference:} I checked the proof in this \href{https://math.stackexchange.com/questions/1266578/show-that-exp-t4-is-not-a-characteristic-function?noredirect=1&lq=1}{Mathexchange post}.
    
    
    \item[(c)] Not a characteristic function by noticing that $ \sin(t)\big|_{t=0}=0\neq 1 $, thus does not satisfy the condition of Bochner's theorem.
    \item[(d)] Note that $ t\mapsto \cos(t) $ has inverse Fourier transform as follows:
    \begin{align*}
        \cos(t)\risingdotseq & \dfrac{ 1 }{ 2\pi } \int _\mathbb{R} \cos(t)e^{-ixt}  \,\mathrm{d}t =\dfrac{ 1 }{ 2\pi } \int _\mathbb{R} \dfrac{ e^{it}+e^{-it} }{ 2 }e^{-ixt}  \,\mathrm{d}t =\dfrac{ 1 }{ 2 }(\delta (x+1)+\delta (x-1))
        \sim \mathrm{ Unif }\{\pm 1\} 
    \end{align*}
    a uniform distribution on $ \{\pm 1\} $.
    \item[(e)] Note that $ t\mapsto \dfrac{ 1+\cos(t) }{ 2 } $ has inverse Fourier transform as follows:
    \begin{align*}
        \dfrac{ 1+\cos(t) }{ 2 }\risingdotseq & \dfrac{ 1 }{ 2\pi } \int _\mathbb{R} \dfrac{ 1+\cos(t) }{ 2 }e^{-ixt}  \,\mathrm{d}t= \dfrac{ 1 }{ 2\pi } \int _\mathbb{R} \dfrac{ 1+\dfrac{ e^{it}+e^{-it} }{ 2 } }{ 2 }e^{-ixt}  \,\mathrm{d}t= \dfrac{ 1 }{ 4 }\big( \delta (x-1)+\delta (x+1) + 2\delta (x) \big)
    \end{align*}
    which is a discrete r.v. with $ P(X=1)=P(X=-1)=1/4 $ and $ P(X=0)=1/2 $.
\end{itemize}

    

\section{Die roll plus uniform}\label{sec:die_roll_plus_uniform}



\begin{itemize}[topsep=2pt,itemsep=0pt]
    \item[(a)] For a fair die $ X $, we have
    \begin{align*}
         X\sim \mathrm{ Unif } \{1,2,3,4,5,6\} \fallingdotseq \dfrac{ 1 }{ 6 }\sum_{k=1}^6 e^{ i k t}
    \end{align*}
    \item[(b)] For $ Y $ uniform on $ [0,1] $, we have
    \begin{align*}
        Y\sim \mathrm{ Unif } [0,1] \fallingdotseq \dfrac{ e^{i t}-1 }{ it }  
    \end{align*}
    \item[(c)] For $ Z=X+Y $, we have
    \begin{align*}
        X+Y\fallingdotseq \dfrac{ 1 }{ 6 }\sum_{k=1}^6 e^{ i k t} \cdot \dfrac{ e^{i t}-1 }{ it } =\dfrac{ 1 }{ 6 }\sum_{k=1}^6 \dfrac{ e^{i (k+1) t}-e^{i k t} }{ it } =\dfrac{ e^{i 7 t}-e^{i t} }{ 6it } \risingdotseq \mathrm{ Unif } [1,7] 
    \end{align*}
    \item[(d)] For $ W=X-Y $, we have
    \begin{align*}
        X-Y\fallingdotseq \dfrac{ 1 }{ 6 }\sum_{k=1}^6 e^{ i k t} \cdot \overline{\dfrac{ e^{i t}-1 }{ it }} =& \dfrac{ 1 }{ 6 }\sum_{k=1}^6 e^{ i k t} \cdot \dfrac{ e^{-i t}-1 }{ -it } = \dfrac{ 1 }{ 6 }\sum_{k=1}^6 \dfrac{ e^{i (k-1) t}-e^{i k t} }{ -it } = \dfrac{ e^{i 6 t} -1 }{ 6it } \risingdotseq \mathrm{ Unif }[0,6]
    \end{align*}
\end{itemize}

    
\section{Difference of two i.i.d. random variables}

Note that for two i.i.d. r.v. $ X,Y $ with characteristic function $ \phi(t) $, we have
\begin{align*}
    X-Y\fallingdotseq \phi(t)\cdot \overline{\phi(t)} =|\phi(t)|^2 \geq 0
\end{align*}
however for uniform distribution $ Z\sim \mathrm{ Unif }[-1,1]  $ we know that
\begin{align*}
    Z\sim \mathrm{ Unif }[-1,1] \fallingdotseq \dfrac{ \sin t }{ t } \text{ is not non-negative}
\end{align*}
which yields contradiction. Thus $ Z $ cannot be represented as the difference of two i.i.d. r.v.s.


\section{Trigonometric identities}

We define the following:
\begin{align*}
    X_n\sim& \mathrm{ Unif }\{\pm \dfrac{ 1 }{ 2^n }\} \fallingdotseq \dfrac{ 1 }{ 2 }\big( e^{i/2^n t}+e^{-i/2^n t} \big)   =\cos(\dfrac{ t }{ 2^n })\\
    Y_n:=& \sum_{i=1}^n X_i \sim \mathrm{ Unif }\{ \dfrac{ k }{ 2^n }  \}_{k=-2^n+1}^{2^n-1}\\ 
    Z\sim & \mathrm{ Unif }[-1,1] \fallingdotseq \dfrac{ \sin t }{ t }
\end{align*}

\begin{itemize}[topsep=2pt,itemsep=0pt]
    \item We note that $ Z\mathop{ = }\limits^{\mathrm{ d } } \mathrm{ Unif }(-\dfrac{ 1 }{ 2 }, \dfrac{ 1 }{ 2 }  ) + X_1  $ (which is easy to verify following the same proof as in Exercise. \ref{sec:die_roll_plus_uniform}). We thus have
    \begin{align*}
        &\dfrac{ \sin t }{ t }\risingdotseq Z \mathop{ = }\limits^{\mathrm{ d } } \mathrm{ Unif }(-\dfrac{ 1 }{ 2 }, \dfrac{ 1 }{ 2 }  ) + X_1  \fallingdotseq \dfrac{ \sin(t/2) }{ t/2 } \cdot \cos(t/2) \\
         \Rightarrow & \dfrac{ \sin t }{ t }=\dfrac{ \sin(t/2) }{ t/2 } \cdot \cos(t/2) 
    \end{align*}
    \item We further have $ Y_n \xrightarrow[]{\mathrm{d}} Z $ by noticing the following:
    \begin{align*}
        \forall \xi \in [-1,1]:\,\mathbb{P}\left( Y_n \geq \xi  \right)  \mathop{ \rightarrow }\limits_{n\to\infty }\mathbb{P}\left( Z \geq \xi  \right)  .
    \end{align*}
    Thus we can conclude convergence in characteristic functions:
    \begin{align*}
        &\prod_{i=1}^n \cos(\dfrac{ t }{ 2^i })\risingdotseq \sum_{i=1}^n X_i = Y_n \xrightarrow[]{\mathrm{d}} Z \fallingdotseq \dfrac{ \sin t }{ t }\\
         \Rightarrow & \dfrac{ \sin t }{ t } = \prod_{i=1}^\infty \cos(\dfrac{ t }{ 2^i }),\quad \forall t\in \mathbb{R}.
    \end{align*}
\end{itemize}

    

\section{Poisson approximation to the binomial; exponential approximation to the geometric}
(This time we prove them "by hand")

\subsection{Poisson approximation to the Binomial distribution}

We have $ \forall k\in \mathbb{N}^+: $
\begin{align*}
    \mathbb{P}\left( \mathrm{ Bin }(n,p_n) = k \right) =&  \binom{n}{k} p_n^k (1-p_n)^{n-k} \\
    =&  \dfrac{ 1 }{ k! }\dfrac{ n! }{ (n-k)! }\cdot \dfrac{ 1 }{ n^n } (np_n)^k (n-np_n)^{n-k} \\
    =& \dfrac{ 1 }{ k! }(n^k + o(n^k))\cdot \dfrac{ 1 }{ n^n } (\lambda +o(1))^k (n-\lambda +o(1))^{n-k} \\
    =& \dfrac{ (\lambda +o(1))^k }{ k! }\dfrac{ (n^k + o(n^k))\cdot(n-\lambda +o(1))^{n-k}  }{ n^n } \\
    =&\dfrac{ (\lambda +o(1))^k }{ k! }\big(1 - \dfrac{ \lambda  }{ n } + o(n^{-1}) \big)^n\\
    \xrightarrow[]{n\to\infty} & \dfrac{ \lambda^k }{ k! }e^{-\lambda }\sim \mathrm{ Poi }(\lambda ) 
\end{align*}
Thus we have proved the Poisson approximation to the Binomial distribution.


\subsection{Exponential approximation to the Geometric distribution}
We have $ \forall t \geq 0$ that 
\begin{align*}
    \mathbb{P}\left( pX_p > t \right) =& \mathbb{P}\left( \mathrm{ Geom }(p)  > t/p \right)= \sum_{k= \left\lceil t/p\right\rceil } ^\infty (1-p)^{k-1}p=(1-p)^{\left\lfloor t/p \right\rfloor } 
\end{align*}

Note that
\begin{align*}
    (1-p)^{t/p} \xrightarrow[]{p\to 0} e^{-t},\quad (1-p)^{t/p-1} \xrightarrow[]{p\to 0} e^{-t}
\end{align*}
we have
\begin{align*}
    \mathbb{P}\left( pX_p > t \right) =&(1-p)^{\left\lfloor t/p \right\rfloor } \xrightarrow[]{p\to 0} e^{-t} = \mathbb{P}\left( \mathrm{ Exp }(1) > t \right)
\end{align*}

Thus we have proved the Exponential approximation to the Geometric distribution.























\end{document}