\documentclass[11pt,a4paper]{article}
%以下为所使用的宏包
\usepackage{ulem}%下划线
\usepackage{amsmath,amsfonts,amssymb,amsthm,amsbsy}%数学符号
\usepackage{graphicx}%插入图片
\usepackage{booktabs}%三线表
%\usepackage{indentfirst}%首行缩进
\usepackage{tikz}%作图
\usepackage{appendix}%附录
\usepackage{array}%多行公式/数组
\usepackage{makecell}%表格缩并
\usepackage{siunitx}%SI单位--\SI{number}{unit}
\usepackage{mathrsfs}%数学字体
\usepackage{enumitem}%列表间距
\usepackage{multirow}%列表横向合并单元格
\usepackage[colorlinks,linkcolor=red,anchorcolor=blue,citecolor=green]{hyperref}%超链接引用
\usepackage{float}%图片、表格位置排版
\usepackage{pict2e,keyval,fp,diagbox}%带有斜线的表格
\usepackage{fancyvrb,listings}%设置代码插入环境
\usepackage{minted}%代码环境设置
\usepackage{fontspec}%字体设置
\usepackage{color,xcolor}%颜色设置
\usepackage{titlesec} %自定义标题格式
\usepackage{tabularx}%列表扩展
\usepackage{authblk}%titlepage作者信息
\usepackage{nicematrix}%更好的矩阵标定
\usepackage{fbox}%更多浮动体盒子



%以下是页边距设置
\usepackage[left=0.5in,right=0.5in,top=0.81in,bottom=0.8in]{geometry}

%以下是段行设置
\linespread{1.4}%行距
\setlength{\parskip}{0.1\baselineskip}%段距
\setlength{\parindent}{2em}%缩进


%其他设置
\numberwithin{equation}{section}%公式按照章节编号
\newenvironment{point}{\raggedright$\blacktriangleright$}{}
\newenvironment{algorithm}[1]{\vspace{12pt} \hrule\hrule \vspace{3pt} \noindent\textbf{\color[HTML]{E63F00}Algorithm } \,\textit{#1} \vspace{3pt} \hrule\vspace{6pt}}{\vspace{6pt}\hrule\hrule \vspace{12pt}} % 算法伪代码格式环境


%代码环境\lst设置
\definecolor{CodeBlue}{HTML}{268BD2}
\definecolor{CodeBlue2}{HTML}{0000CD}
\definecolor{CodeGreen}{HTML}{2AA1A2}
\definecolor{CodeRed}{HTML}{CB4B16}
\definecolor{CodeYellow}{HTML}{B58900}
\definecolor{CodePurPle}{HTML}{D33682}
\definecolor{CodeGreen2}{HTML}{859900}
\lstset{
    basicstyle=\tt,%字体设置
    numbers=left, %设置行号位置
    numberstyle=\tiny\color{black}, %设置行号大小
    keywordstyle=\color{black}, %设置关键字颜色
    stringstyle=\color{CodeRed}, %设置字符串颜色
    commentstyle=\color{CodeGreen}, %设置注释颜色
    frame=single, %设置边框格式
    escapeinside=`, %逃逸字符(1左面的键),用于显示中文
    %breaklines, %自动折行
    extendedchars=false, %解决代码跨页时,章节标题,页眉等汉字不显示的问题
    xleftmargin=2em,xrightmargin=2em, aboveskip=1em, %设置边距
    tabsize=4, %设置tab空格数
    showspaces=false, %不显示空格
    emph={TRUE,FALSE,NULL,NAN,NA,<-,},emphstyle=\color{CodeBlue2}, %其他高亮}
}


%节标题格式设置
\titleformat{\section}[block]{\large\bfseries}{Exercise \arabic{section}}{1em}{}[]
\titleformat{\subsection}[block]{}{    \arabic{section}.(\alph{subsection})}{1em}{}[]
% \titleformat{\subsubsection}[block]{\normalsize\bfseries}{    \arabic{subsection}-\alph{subsubsection}}{1em}{}[]
% \titleformat{\paragraph}[block]{\small\bfseries}{[\arabic{paragraph}]}{1em}{}[]


% \titleformat{\sectioncommand}[shape]{format}{title-label}{sep}{before-title}[after-title]



% 中文字号
% 初号42pt, 小初36pt, 一号26pt, 小一24pt, 二号22pt, 小二18pt, 三号16pt, 小三15pt, 四号14pt, 小四12pt, 五号10.5pt, 小五9pt



\begin{document}

\begin{center}\thispagestyle{plain}

{\LARGE\textbf{STAT 430-1, Fall 2024}}

{\Large\textbf{HW2}}

Tuorui Peng\footnote{TuoruiPeng2028@u.northwestern.edu}
\end{center}

\thispagestyle{myheadings}\markright{Compiled using \LaTeX}
\pagestyle{myheadings}\markright{Tuorui Peng}

% show only sections level in contents
\setcounter{tocdepth}{1}
\tableofcontents


  

\section{Atoms}


Denote by $ K_n $ the set of atoms with mass greater than $ 1/n $, then we have
\begin{align*}
     \mu (E)  \geq \mu (K_n) \geq \dfrac{ \left\vert K_n \right\vert  }{ n }  \Rightarrow \left\vert K_n \right\vert\leq n\mu (E)
\end{align*}
which means that $ K_n $ is finite for any $ n\in\mathbb{N}^+ $. And we also note that $ \{K_n\}_{n\in\mathbb{N}^+} $ is an increasing sequence of sets. By definition of stones, we have the set of all stones being $ \mathcal{S}=\bigcup_{n=1}^{\infty}K_n = \uplus_{n=1}^{\infty}K_n\backslash K_{n-1} $. i.e. $ \{K_n\backslash K_{n-1}\}_{n\in\mathbb{N}^+} $ is a partition of $ \mathcal{S} $.

\textbf{If} $ S $ is uncountable, then there exists $ n_0\in\mathbb{N}^+ $ such that $ K_{n_0}\backslash K_{n_0-1} $ is uncountable, an thus $ K_{n_0} = \cup_{k=1}^{n_0} K_{n_0}\backslash K_{n_0-1} $ is also uncountable, which contradicts the fact that $ \left\vert K_{n_0} \right\vert \leq n \mu (E) <\infty $.

Thus $ S $ is countable.


\section{Fatou's Lemma}

Note that $ \{\inf_{m\geq n}f_m\}_{n\in\mathbb{N}} $ is a monotone increasing sequence of functions. Using monotone convergence theorem, we have
\begin{align*}
    \mu \left(\mathop{ \lim\inf }\limits_{n\to\infty } f_n \right)  = & \mu \left( \mathop{ \lim }\limits_{n\to\infty } \inf_{m\geq n} f_m \right)\\
    = & \lim_{n\to\infty } \mu \left( \inf_{m\geq n} f_m \right)\\
    \leq & \lim_{n\to\infty } \inf_{m\geq n} \mu (f_m)\\
    = & \mathop{ \lim\inf  }\limits_{n\to\infty} \mu (f_n) 
\end{align*}

\section{Dominated convergence theorem}

From above, we already have $ \mu (f) \leq \liminf_{n\to\infty} \mu (f_n) $. Now it suffice to show the other direction.

Note that we have dominance condition $ -g \leq f_n \leq g $, $ \forall n\in \mathbb{N}^+ $, thus we can consider the following sequence of functions: $ \{ g-f_n \}_{n\in\mathbb{N}^+} $, which has
\begin{align*}
    0\leq g-f_n \leq 2g 
\end{align*}
thus is still integrable. By Fatou's lemma, we have
\begin{align*}
    \mu (g-f) \leq \liminf_{n\to\infty} \mu (g-f_n)  \Rightarrow \mu (f) \geq \limsup_{n\to\infty} \mu (f_n)
\end{align*}
which concludes the proof.

\section{On interchanging limits and integration}

\begin{itemize}[topsep=2pt,itemsep=0pt]
    \item \textbf{Left hand side}: we have
    \begin{align*}
        \lim_{\lambda \to 0 }\int_0^\infty f_\lambda (x)\,\mathrm{d}x = \lim_{\lambda \to 0 }\int_0^\infty \lambda e^{-\lambda x}\,\mathrm{d}x = \lim_{\lambda \to 0 }1 = 1 
    \end{align*}
    \item \textbf{Right hand side}: we have
    \begin{align*}
        \int_0^\infty \lim_{\lambda \to 0 }f_\lambda (x)\,\mathrm{d}x = \int_0^\infty \lim_{\lambda \to 0 }\lambda e^{-\lambda x}\,\mathrm{d}x = \int_0^\infty 0\,\mathrm{d}x = 0 
    \end{align*}
\end{itemize}

Thus the two sides are not equal, and the limit cannot be interchanged with the integral. 

To explain why this happens, we illustrate the following that there is no dominating function for the sequence of functions $ \{f_\lambda (x)\}_{\lambda>0} $: for any $ x\in\mathbb{R}^+ $, and any $ \varepsilon \in (0,1) $, we can find $ \lambda_\varepsilon  $ s.t. $ \lambda_\varepsilon  e^{-\lambda_\varepsilon  x} = \varepsilon $, thus to dominate the sequence, we need to have $ g(x) \geq \varepsilon $, $ \forall x\in\mathbb{R}^+ $, which makes $ g(x) $ un-integrable. Thus there is no dominating function for the sequence of functions, and the condition of Lebesgue's dominated convergence theorem is not satisfied.

\section{On interchanging integrals}

\begin{itemize}[topsep=2pt,itemsep=0pt]
    \item We have
    \begin{align*}
        I =& \int_0^\infty \,\mathrm{d}x\int_0^\infty \,\mathrm{d}y f(x,y) = \int_0^\infty \,\mathrm{d}x \begin{cases}
            x-1, & x\in [0,1]\\
            0, & x\in (1,\infty)
        \end{cases}\\
        =& \int_0^1 \,\mathrm{d}x (x-1) = -\dfrac{1}{2}
    \end{align*}

    \item We have
    \begin{align*}
        J=& \int_0^\infty \,\mathrm{d}y\int_0^\infty \,\mathrm{d}x f(x,y) = \int_0^\infty \,\mathrm{d}y \begin{cases}
            1-y, & y\in [0,1]\\
            0, & y\in (1,\infty) 
        \end{cases}\\
        =& \int_0^1 \,\mathrm{d}y (1-y) = \dfrac{1}{2}
    \end{align*}
\end{itemize}

i.e. we got $ I\neq J $, and the integrals cannot be interchanged. We can explain this by noticing that the condition of Fubini's theorem is not satisfied: the function $ f(x,y) $ is not integrable on $ \mathbb{R}^2 $:
\begin{align*}
    \int_0^\infty \,\mathrm{d}x\int_0^\infty \,\mathrm{d}y \left\vert f(x,y) \right\vert =\infty.
\end{align*}



    


    





















\end{document}